\documentclass{article}
\usepackage{amsmath, amsfonts, amssymb}
\begin{document}
	\begin{center}
		\Large \textbf{Artificial Intelligence Course Project: The 8- and The 15- Puzzle}
	\end{center}
	\begin{center}
		\textit{Sydney Garcia, Tanmay Inamdar, Arjun Viswanathan}
	\end{center}

	\section{Hypothesis}
		This is the report for the semester project for CS:4420 Artificial Intelligence. The project is divided into 2 parts. In the first part, we propose to compare the application of the A* search algorithm on the 8-puzzle problem using 3 different heuristics functions: Manhattan distance, misplaced tiles, and max-swap. We use the number of expanded nodes, the cost of the solution, the running time, and the effective branching factor as metrics for this comparison. In the second part, we implement the 15-puzzle problem using the IDA* search strategies for testing. \par 
		
		\textbf{Keywords: } N-Puzzle, A*, IDA*, Heuristics
		
	\section{Report}
		
		The Manhattan distance heuristic for the n-puzzle problem calculates, given a state, the sum of the Manhattan distances of each tile from its respective position in the final state. The max-swap heuristic function calculates the number of tile swaps with the blank space required to go from start state. The misplaced tile heuristic function counts the number of tiles that are out of place compared to the goal state, not including the blank tile. The Manhattan distance and misplaced tile heuristics are both stated as being admissible heuristics in our class text. All three of the hueristics we used were also cited to be admissible in the reference text [MP89]. 
		
	\section{Acknowledgements}
	
	\section{References}
\end{document}
